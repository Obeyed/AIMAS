\subsection{Online and Offline Planning}

In this section the aim is to discuss the differences between online and offline planning, where we worked with online planning.

Both online and offline planning are applicable in the domain of the project.
The domain is deterministic, static, and fully observable and we know that there are no unforeseen side-effects such as some event that can change the state of the level.
The biggest challenge of offline planning is the combinatorial explosion/branching factor for planning for all possible contingencies for an entire plan.
All possible conflicts must be found and resolved before any action is performed.

We can consider online planning as performing small steps in aim of solving a grand goal.
We believe this is a more natural approach to solving the programming project.
Unable to look into the future, the online planning approach will most likely not find an optimal plan.
On the other hand the online approach gives us the advantage of performing actions early on.

Furthermore, with online planning the client can plan for unseen future effects of past actions.
Note, we assumed that the domain was deterministic and fully observable for offline planning to be applicable.
By monitoring the execution of actions the client can keep and updated perception of the world in which the agents are performing actions.
The fact that, for every step performed, the grand goal is closer to be solved.
By dividing the grand goal into smaller sub-goals and solving obstacles as they appear for the sub-goals, we do not experience the dramatic branching factor, compared to an offline approach.

We found that online planning was more intuitive and simpler to develop.
We believe that our solution is simple because of the simple nature of online planning.
We monitor the server's response and from that we construct and achieve the next goal in the line.
