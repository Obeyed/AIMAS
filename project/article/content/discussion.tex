\section{Discussion}
\label{sec:discussion}
%Strengths: Goal-prio (yay),
%Only 1 agent at a time: Weakness in solution length but strength in performance
%TODO: Mention possible impovement on Multi Agent tracks, gained from sending all commands at once. Include remarks on possible issues with this approach as well. I.e. in a level with 2 angents and 100 steps, the acutal steps could be 50, provided that no new issues arises.

This section presents some of the pros and cons of our client.
For instance we will discuss the con of only moving one agent at a time and the difference between online and offline planning.

\subsection{On Multi-agent Movement}
Our current solution only sends commands to one agent at a time.
This increases the length of results in multi-agent levels but it also give us better performance.
By performance we mean the time spent to find a solution (from starting the client to the server responds with \texttt{success}).
By only moving one agent at a time we do not have to look for conflicts between the agents' paths but simply move obstacles.
The simplicity of our client was advantageous while competing against the other group's more complicated solutions.
It was clear that we could out-perform them w.r.t. time spent on finding a solution.

Of course, if the client could move multiple agents simultanouesly, this would have been a great improvement w.r.t. number of actions performed.
For example consider \cref{fig:intuition example} again, if all agents moved at the same time then all the blocking boxes would be moved out of the way simultaneously.
But by moving all agents simultaneously we would risk getting conflicts if two or more agents' paths crossed.
To solve these conflicts we could either do offline planning and make a complete plan before execution.
This would hurt our solving times on multi-agent levels since we would have to cross check all paths.

We could also try to solve the conflicts as they come in our online planning.
This way we would solve conflicts as we go but we risk live-locks if two agents get stuck in each others paths.
We would be forced to handle these challenges, which would most likely decrease our client's performance.
Another way is to implement it into our current online solution by making dependent paths move simultaneously if their paths do not cross.
Otherwise, the path without dependencies should move first.
This way it would most likely only hurt our performance a little, but the number of actions would not be improved a lot and only on some levels.

%\emph{What are the strengths and weaknesses of your solution? Why did you choose your particular solution method, and did it work out as expected? If it didn’t work as expected, what should have been different? Here you can also include a deeper comparative analysis if you have implemented different methods/clients. Which worked best? Can you get better than that? How?}
%
We decided to do goal prioritisation over agent prioritisation because we wanted to solve the inner most goals if goals would block for each other.
Our goal prioritisation makes sure that we in most cases do not solve a goal if it will block for a unsolved goal.
If we had prioritisation on agents we could have gotten shorter solutions in some levels where an agent start close to a box and goal but it chooses another goal that is further away because that goal has a higher priority.
Our solution gives us less conflicts and less moving of boxes on solve goals.


\subsection{Online and Offline Planning}

In this section the aim is to discuss the differences between online and offline planning, where we worked with online planning.

Both online and offline planning are applicable in the domain of the project.
The domain is deterministic, static, and fully observable and we know that there are no unforeseen side-effects such as some event that can change the state of the level.
The biggest challenge of offline planning is the combinatorial explosion/branching factor for planning for all possible contingencies for an entire plan.
All possible conflicts must be found and resolved before any action is performed.

We can consider online planning as performing small steps in aim of solving a grand goal.
We believe this is a more natural approach to solving the programming project.
Unable to look into the future, the online planning approach will most likely not find an optimal plan.
On the other hand the online approach gives us the advantage of performing actions early on.

Furthermore, with online planning the client can plan for unseen future effects of past actions.
Note, we assumed that the domain was deterministic and fully observable for offline planning to be applicable.
By monitoring the execution of actions the client can keep and updated perception of the world in which the agents are performing actions.
The fact that, for every step performed, the grand goal is closer to be solved.
By dividing the grand goal into smaller sub-goals and solving obstacles as they appear for the sub-goals, we do not experience the dramatic branching factor, compared to an offline approach.

We found that online planning was more intuitive and simpler to develop.
We believe that our solution is simple because of the simple nature of online planning.
We monitor the server's response and from that we construct and achieve the next goal in the line.

\subsection{Centralised and Decentralised Planning}

Multibody vs. multiagent.

conflict resolution amongst agents.
social laws.
live lock detection.

communication, coordination.

