\section{Discussion}
\label{sec:discussion}
%Strengths: Goal-prio (yay),
%Only 1 agent at a time: Weakness in solution length but strength in performance
%TODO: Mention possible impovement on Multi Agent tracks, gained from sending all commands at once. Include remarks on possible issues with this approach as well. I.e. in a level with 2 angents and 100 steps, the acutal steps could be 50, provided that no new issues arises.
Our current solution only sends commands to one agent at a time.
This increases the length of results in multi-agent levels but it also give us better performance.
By performance we mean the time spent to find a solution (from starting the client to the server responds with \texttt{success}).
By only moving one agent at a time we do not have to look for conflicts between the agents paths but simply move obstacles.
This strategy was very efficient to get the fastest times.
When compared to the other clients the was clear our solution was the fastest by far in multi-agent levels.

There were potential improvement on the result lengths in many multi-agent levels if we moved all agents simultaneously.
E.g. if all agents moved at the same time in \cref{fig:intuition example} all the blocking boxes would be moved out of the way simultaneously.
But by moving all agents simultaneously we would risk getting conflicts if two or more agents paths crossed.
To solve these conflicts we could either do offline planning and make a complete plan before execution.
This would hurt our solving times on multi-agent levels since we would have to cross check all paths.

We could also try to solve the conflicts as they come in our online planning.
This way we would solve conflicts as we go but we could risk deadlocks or even longer paths if two agents got stuck in eachothers paths.
We could make functions that could solve these problems but they would most likely decrease our performance.
Another way is to implement it into our current online solution by making dependent paths move simultaneously if their paths do not cross.
Else the path without dependencies should move first.
This way would most likely only hurt our performance a little but the result lengths would not be improved a lot and only on some levels.

%\emph{What are the strengths and weaknesses of your solution? Why did you choose your particular solution method, and did it work out as expected? If it didn’t work as expected, what should have been different? Here you can also include a deeper comparative analysis if you have implemented different methods/clients. Which worked best? Can you get better than that? How?}

\subsection{Online and Offline Planning}

Unable to look into the future because we solve sub-goals and don't build a plan for solving the entire level.

Offline planning is applicable for this domain because it is deterministic, static and fully observable.

Biggest challenge for offline planning is the combinatorial explosion/branching factor for planning for all possible contingencies for an entire plan.

More naturally online planning (and replanning) aims to achieve sub-goals that will eventually lead to a solution to the grand goal.

Furthermore, with online planning the client can plan for unseen future effects of past actions.
Note, we assumed that the domain was deterministic and fully observable for offline planning to be applicable.
By monitoring the execution of actions the client can keep and updated perception of the world in which the agents are performing actions.


\subsection{Centralised and Decentralised Planning}

Our client incorporates a single-agent multibody planner, which plans to achieve goals in a prioritised order.
Compared to a multi-agent planner, our client will never encounter conflicts with other agent's plans that need to be resolved.
On the other hand, if multiple agents could have performed actions simultaneously now they must wait for the other to finish, and thus the number of actions performed will automatically be higher.

If we consider that the planner could handle multiple agents simultaneously, then the client must be able to handle conflicts with other agents.
This would for instance be in the form of social laws, where agents consider each others identifier to decide which agent must make room for the other.
Agent's could also communicate their intentions to each other, and thereby coordinate conflict free paths.
Moreover, it could also be possible to perform some kind of agent prioritisation based on the agent's identifier or the path which the respective agent is currently moving along.
As we presented in \cref{sec:related work}, some of these ideas have been proposed in related work.

Furthermore, if multiple agents move simultaneously then it could happen that their movement would be caught in a live lock, i.e. they move in an endless loop trying to move past each other.

