\section{Introduction}
\label{sec:introduction}

%\emph{What problem are you trying to solve? Why is this important? What are the main results/accomplishments/highlights of the paper. Since the reader is assumed to be familiar with the details and goal of the programming project, this section will probably be fairly short. But you can use it to briefly describe the type of solution you have chosen and why.}

This paper documents and presents our solution to the programming project.
Our first attempt to solve the project was based on a partial-order planner for atomic plans.
Our thoughts were that we could use the planner to re-plan if conflicts arose.
However, we abandoned this solution because it would simply give us a backwards breadth-first search, and we knew we could develop a better solution.
We used the Python language to develop our source code.

Our solution is a single-agent online centralised multibody solution, where we attempt to prioritise the order in which goals are fulfilled.
Paths were constructed in a hierarchical fashion using the weighted A* search algorithm in a relaxed domain.
%We present our own goal prioritisation solution and experiment with different types of prioritisations.
%Furthermore, we used the A* search algorithm to find movement paths, for which we incorporated the ignore preconditions heuristics to find paths in a relaxed domain.

\subsection{Problem Statement}

The aim of the project is to complete a level by moving boxes to their respective goal positions.
Challenges such as coordinating agent movement, resolving conflicting movement, communication between agents, moving obstacles from a path to reach a desired destination, etc. must be handled properly for the level to be solved.
If they are left unresolved the client may never be able to find a solution.

Similar automated clients could be useful in many scenarios where repetitive work is needed.
This could for instance be in a hospital where automated agents would be responsible for moving empty hospital beds to a destination where they are needed.


\section{Background}
\label{sec:background}

In this section we will breifly outline the different theories and methods, that act as the basis of our client solution. Some of the techniques outlined in this section rely on either theories taught in the course or found in external literature, while others are of our own invention. The specific implementations of these, can be found in Section \ref{sec:methods}

\subsection{Goal Ordering / Prioritisation}
Of the different techniques used to increase the performance of our client, on of the most important 
\\\\


\emph{Briefly present the theories that you work relies on. Note that you are allowed to expect the reader to be familiar with everything presented in the course curriculum, but you should at least mention which of these theories you are using and make suitable references. Even for the theories in the course curriculum it might also be necessary to settle the notational conventions, as these sometimes differ between references. If you use theories and ideas outside the course curriculum, you should explain them in a bit more detail, and of course also make suitable references.}

\subsection{Related Work}
\label{sec:related work}
Our first plan was to make a partial-order planner for multiple agents on an atomic level.
Knoblock examined how a partial-order planner would be able to construct parallel plans~\cite{knoblock1994generating}.

Ephrati had a similar approach to ours in 1994 where they suggested multi-agent planning with heuristics~\cite{ephrati1994divide}.
They divided the goal in sub-goals and made sub-plans that were merged to a global plan.
Their results showed that dividing into sub-goals and making the agents work in parallel can reduce the total elapsed time for the actual planning.
Similarly, we used a high-level planner to divide goals into independent sub-goals and solve them independently.

Our goal prioritisation technique is an original contribution, however we have found papers that also use prioritised planning.
For instance, \cite{van2005prioritized,bennewitz2001optimizing,erdmann1987multiple} present planning techniques where multiple objects must be moved, and the movement plans are constructed by prioritising either the paths or the objects to be moved.
This differs from our goal priotization in that they consider the object or the path, while we consider in which order to fulfill a goal.
Furthermore, prioritised planning is not a new concept.
Both in real life and in games plans must be prioritised, e.g. to stay alive in a game, the player or the game's AI agent must prioritise fleeing from danger or advancing on an enemy.~\cite{orkin2006three}


%\emph{Has this been done before? What is the closest related research? How does your work differ? Related work is sometimes integrated into the introduction or background, and sometimes it is made a separate section towards the end of the paper.  To make the related work section, you will be required to do some literature search to see if you can find papers that use similar methods (or combinations of methods) on similar types of problems. A piece of related work could for instance be if someone wrote a paper on using similar methods for the Sokoban domain. Sometimes it can be hard to find related work, but you should do your best. Use e.g. Google and Google Scholar (if a certain relevant paper is licensed, try to download it via findit.dtu.dk).}


