\subsection{Representing a Level}
\label{sec:representing a level}

\subsubsection{Cell of a Level}

We represent levels as grids of cells, where each cell can hold a wall, goal, agent, or a box.
Otherwise it is a free cell.
Each cell is uniquely referenced with coordinates, that define the x and y coordinates w.r.t. the top left corner.
Thus, the first cell is referenced as $(0,0)$ and is in the top left corner of the level.

\subsubsection{The Grid}

A new grid object is instantiated with the level parsed from the server.
The grid instantiation process takes care of assigning default colors to uncolored objects.
The grid keeps track of the following; all object's positions (their cells) and colors, goals and their identifiers, the boxes and the agents.
It knows which color boxes an agent can move.
Furthermore, the grid representation must have the following functionality
%
\begin{itemize}
  \item for a given cell, return its neighbouring cells
  \item for a given cell, is it possible to perform a swapping move
  \item for all goals, compute their priority (see \cref{methods:goal_ordering})
  \item for every server response (with every action we send), update the grid so it is up-to-date
\end{itemize}
%
The neighbours of a cell may be defined to filter results w.r.t. some criterion.
Such a criterion could for instance be that we only want free cell (no agent or box at that cell).
This method is used for finding possible movement from a given cell.
Unlike in our goal prioritisation, where a cell has 8 neighbours, the diagonal cells are not considered neighbour cells here and thus a cell only has 4 neighbours.
%Moreover, a cell's neighbours are the cells below, above, right of, and left of it (not diagonal cells).

We define a swapping move as swapping the position of an agent and a box it is moving.
This could either be a pull-push combination or vice versa.
The idea is to check if a given cells has enough neighbours for a swap to be performed, i.e. the cell must have at least three neighbours (because then we at least have a \texttt{T}-cell-combination, where a swap is possible).

The client performs online planning, i.e. an entire complete plan is not computed before performing actions.
Because of this the client must keep track of all actions and if they succeed.
The server's response must be used to keep the level representation up-to-date.

\subsubsection{Movable Objects}

We define agents and boxes as movable objects.
They have a position (a cell), some color, and an identifier that uniquely defines them in the level.
Such an object can thus also be moved, which changes the position at which it is for that time step.

\subsubsection{Updating the Representation}

For every action sent to the server, we make sure to update the client's perception of the current environment.
Positions of agents and boxes are thus kept up-to-date while actions are performed.
