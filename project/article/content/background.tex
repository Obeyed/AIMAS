\section{Background}
\label{sec:background}

In this section we will breifly outline the different theories and methods, that act as the basis of our client solution. Some of the techniques outlined in this section rely on either theories taught in the course or found in external literature, while others are of our own invention. A descriptipn of the algorithmic details of these, can be found in Section \ref{sec:methods}

We assume the reader is familiar with the course curriculum.
Throughout the course we were introduced to many techniques which we drew inspiration from to aid in developing our solution/client.
The client is an online multibody planner, where only one agent moves at any given time.
Plans are constructed iteratively in an hierarchical fashion using a custom A* search algorithm with an admissible heuristic on a relaxed search space.~\cite{russell2009modern,geffner2013concise} 


\subsection{Goal Ordering / Prioritisation}
Of the different techniques used to increase the performance of our client, on of the most important was goal prioritisation. Without goal prioritisation the agent(s) would attempt to move the boxes to their respective goals, but in a non-specific order, often blocking the future path of other blocks or agents. To counter this we needed to enable the client to order the different goals, according to which needed to be fulfilled first. To achieve the goal odering ability we came up with our own teqhnique of scoring each goals according to their neighboring cells. In Section \ref{methods:goal_ordering}, we go into the algorithmic description our goal ordering.



\emph{Briefly present the theories that you work relies on. Note that you are allowed to expect the reader to be familiar with everything presented in the course curriculum, but you should at least mention which of these theories you are using and make suitable references. Even for the theories in the course curriculum it might also be necessary to settle the notational conventions, as these sometimes differ between references. If you use theories and ideas outside the course curriculum, you should explain them in a bit more detail, and of course also make suitable references.}
