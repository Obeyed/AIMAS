\section{Methods}
\label{sec:methods}

%\emph{How does your client work? Describe its overall algorithmic functionality as clearly and precisely as possible, without going into any implementation details.  Describe it in terms of the theories you rely on (as presented in background). Include small, illustrative examples of how the client works. Aim to describe your client suffi- ciently precisely that a reader being an expert in the field would be able to implement a similar solution based on your description (note that this is not an encouragement to describe implementation details, but rather make sure that the overall structure and ideas of your implementation are sufficiently precise).}

This section describes our client's overall algorithmic functionality.
The client is comprised of our goal prioritization technique (\cref{methods:goal_ordering}), the level representation (\cref{sec:representing a level}), the path finding algorithm (\cref{sec:constructing paths and searching}), and the overall planner (\cref{sec:planning on the fly}).
The client receives an incomplete plan from the planner, where the client makes sure to update the plan with the necessary \texttt{NoOp} operations for waiting agents.

Before we begin describing the client's overall functionality, we will give an intuition of how the planner works.
Consider the scenario presented in \cref{fig:intuition example}.
The red agent \textbf{0} must move the red box \textbf{A} to the goal.
Multiple boxes are in the way of the desired plan.
The state of the level in the figure is that the cyan agent \textbf{1} has moved its box out of the way, and so has the yellow agent \textbf{2}.

\begin{figure}[ht!]
  \centering
  \includegraphics[width=.95\columnwidth]{graphics/planner_intuition.png}
  \caption{\label{fig:intuition example}The red agent \textbf{0} must move box \textbf{A} to the goal \textbf{a}.}
\end{figure}

However, the yellow agent \textbf{2} is now itself in the way of the path.
The planner thus informs that agent to move out of the desired path.
This moving of boxes/agents out of the way of the desired plan will continue until no more obstacles are in the way.
Then the red agent \textbf{0} will move the red box \textbf{A} to the goal.

\subsection{Goal Prioritisation Technique}
\label{methods:goal_ordering}

In order to explain how the functionality of our priorotisation works, we will break it down into three separate steps. For all the three steps we use an example set of goal cells, which are all only accessible form a single entrance. 

% Example of cells 
\begin{figure}[h!]
  \centering
  \includegraphics[width=.5\columnwidth]{graphics/ie_level.png}
  \caption{\label{fig:sample}Goal set with single entrance.}
\end{figure}

For the example in \cref{fig:sample} the optimal ordering would be to fulfil the goals in the following order: \textbf{a, c, b, e, f} from first to last. 
To accomplish this, our algorithm provides an ordering score to each goal. 

\subsubsection{Step one}has the purpose of searching through each goal, evaluating its eight neighbouring cells, including the diagonal neighbours. 
For every neighbouring goal-cell, the score of the cell (that is being prioritised) is incremented by 1.
For every free neighbour cell, the score is incremented by 2. 
The checking of neighbouring cells is illustrated with a grid in \cref{fig:grid1}. 
The middle cell is the one currently being checked, while the green represents a neighbour goal. 
A red cell represents an unreachable position (either a wall or out of bounds).

% Example of scoring 
\begin{figure}[ht!]
  \centering
  \includegraphics[width=.5\columnwidth]{graphics/goal_pri_1.png}
  \caption{\label{fig:grid1}Neighbour checking of goal \textbf{a}.}
\end{figure}

The goal \textbf{a} is given a start score of 1. 
Goal \textbf{f} and \textbf{b} are given scores of respectively 3 and 2 as can be seen in \cref{fig:grid2}. 
A blue cell in the grid represents a free neighbour, while a green cell represents a neighbouring goal.
The remaining goals not shown in the figures are all given a score of 2. 

% Example of scoring 
\begin{figure}[h!]
  \centering
  \includegraphics[width=.5\columnwidth]{graphics/goal_pri_3.png}
  \includegraphics[width=.5\columnwidth]{graphics/goal_pri_2.png}
  \caption{\label{fig:grid2}Neighbour checking of goals \textbf{f} \& \textbf{b}.}
\end{figure}

\subsubsection{Step two}is reached after scoring the goals according to their neighbours. 
In this step we create a matrix of size $n \times n$, where $n$ is the number of goals. 
Each column of the matrix will represent a goal cell and the first row holds the ordering scores that were computed in the first step. 

\begin{table}[h!]
  \caption{\label{tab:example_matrix}\centering Example level - Ordering matrix \break Incomplete (left) \& Complete (right)}
	\begin{minipage}{.5\linewidth}
    \centering
    \begin{tabular}{@{}lllll@{}}
		\toprule
		\textbf{f} & \textbf{e} & \textbf{b} & \textbf{c} & \textbf{a} \\ \midrule
		3          & 2          & 2          & 2          & 1          \\ 
		-          & -          & -          & -          & -          \\ 
		-          & -          & -          & -          & -          \\ 
		-          & -          & -          & -          & -          \\ 
		-          & -          & -          & -          & -          \\ \bottomrule
		\end{tabular}
  \end{minipage}%
  \begin{minipage}{.5\linewidth}
    \centering
    \begin{tabular}{@{}lllll@{}}
		\toprule
		\textbf{f} & \textbf{e} & \textbf{b} & \textbf{c} & \textbf{a} \\ \midrule
		3          & 2          & 2          & 2          & 1          \\ 
		4          & 2          & 3          & 2          & 1          \\ 
		5          & 3          & 3          & 2          & 1          \\ 
		6          & 4          & 3          & 2          & 1          \\ \midrule
		\textbf{7} & \textbf{4} & \textbf{3} & \textbf{2} & \textbf{1} \\ \bottomrule
		\end{tabular}
  \end{minipage} 
\end{table}

The resulting matrix can be seen in \cref{tab:example_matrix} (left). From this matrix we see the first ordering scores, beeing filled out, while the rest remains to be computed. The computation of the remaining rows, is caried out by incrementing the scores iteratively. A goal's score is incrementef if:

\begin{itemize} 
\item{The score is the maximum score of all score}
\item{The score is smaller than the previous and next score }
\item{The score is smaller than previous and equal to then next score}
\end{itemize} 
	
\subsubsection{Step three}occurs when all rows have been iterate through and the matrix is filled out, as seen in \cref{tab:example_matrix} (right). Ideally, this will leave the last row with unique scores, that can be used to order which goals to fulfill first. The third step is simply to order the goals from the least to the heighest scoring goal, such that in our example case, the goals will be filled from \textbf{a} to \textbf{f} respectively.
The performance benifits and pitfalls of this prioritisation algorithm, will be discussed in detail in \cref{subsec:disc_goal_ordering}.
%
% In levels where goals are grouped independently of other goal groups, the algorithm will allow for some scores to be the same.
% This signals that some goals can be achieved in an arbitrary order without one having to be filled before the other.
% The last row of the matrix in \cref{tab:example_matrix} (right), is now used to order the goals.



\subsection{Representing a Level}

We use grids to represent levels, where 

\subsection{Searching with A*}

We used A* for finding paths.~\cite{pathfinding2016redblobgames}

\subsubsection{Tie-breaking Heuristics}

We used a tie-broken cross product as a heuristic.~\cite{pathfinding2016redblobgames}

\subsubsection{Ignoring Heuristics}

Furthermore, we used no heuristic to find first free cell to drop boxes or agents.

\subsection{Planning on the Fly}
\label{sec:planning on the fly}

We perform online planning, i.e. we do not create an entire complete plan before executing movement.
Given a constructed plan as described in \cref{sec:constructing paths and searching} the client, that plan may contain positions on which obstacles are in the way.
Given that obstacle the client will create a new sub-plan, which must be achieved before the original plan can continue.
This process may continue for many steps before all obstacles are moved for the original objects to be moved in the desired path.
Note however, that sub-plans of a sub-plan may also occur.

As mentioned in the previous section, each movement has an associated cost, and the agents are to be as independent as possible.
However, independence is sometimes not possible and thus the agents must aid one another.
The planner thus tries to create a sub-plan, where that plan's goal is to move the obstacle.
This is done in an iterative manner, where the planner checks if the current plan/sub-plan is conflict/obstacle free and thus returns a list of desired actions to the client, which then sends actions to the server.

The planner is continuously called with new goals to be achieved until all goals are occupied by a corresponding box.
Note, that a plan is created and executed for a single agent at a time.


