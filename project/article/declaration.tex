%% Skabelon til LiCS-afleveringer

%%%%%%%%%%%%%%%%%%%%%%%%%%%%%%%%%%%%%%%%%%%%%%%%%%%%%%%%%%%%%%%
%% Begynd preamble
%%%%%%%%%%%%%%%%%%%%%%%%%%%%%%%%%%%%%%%%%%%%%%%%%%%%%%%%%%%%%%%
\documentclass[a4paper]{article}

%% Font and input encoding
%% Tillad æøå
\usepackage[T1]{fontenc}
\usepackage[utf8x]{inputenc}
%% Babel (language)
\usepackage[english]{babel} % If you write in English
%\usepackage[danish]{babel} % Hvis du skriver på dansk

%% AMS-Math packages
\usepackage{amsmath}
\usepackage{amssymb}
\usepackage{amsthm}
\newtheorem{theorem}{Theorem}
%% Extra symbols that we almost always need
\usepackage{stmaryrd}
\usepackage{color}
\usepackage{url}
\usepackage{semantic}
%\usepackage{boxproof} % I boxproof.sty
\usepackage{graphicx}
\usepackage{tikz}
\usepackage{subcaption}
%%%%%%%%%%%%%%%%%%%%%%%%%%%%%%
%% Skift sidenumrene ud med X/total (lettere at rette :-)
\usepackage{lastpage}
\makeatletter
\renewcommand{\@oddfoot}{\hfil \thepage{}/\pageref{LastPage} \hfil}
\renewcommand{\@evenfoot}{\hfil \thepage{}/\pageref{LastPage} \hfil}
\makeatother
%%%%%%%%%%%%%%%%%%%%%%%%%%%%%%
\usepackage{listings}
\lstset{
  breaklines=true,
  keepspaces=true,
  frame=ltrb,
  framesep=1pt,
  commentstyle=\color{Brown},
  basicstyle=\ttfamily\footnotesize,
  numbers=left,
  title=\lstname,
  columns=fullflexible,
}
%%%%%%%%%%%%%%%%%%%%%%%%%%%

\title{TheRedDot (02285 AI \& MAS)\\Group Declaration}
\author{
  Elias K. Obeid (s142952) \hspace{10pt}
Jens-Christian Finnerup (s143120)\\
Sune Bartels  (s155994) \hspace{10pt}
Mathias C. Lyngman (s127719)}

\newcommand{\jens}{Finnerup}
\newcommand{\sune}{Bartels}
\newcommand{\superdude}{Lyngman}
\newcommand{\elias}{Obeid}

%%%%%%%%%%%%%%%%%%%%%%%%%%%%%%%%%%%%%%%%%%%%%%%%%%%%%%%%%%%%%%%
%% Slut preamble -- herunder følger selve dokumentet!
%%%%%%%%%%%%%%%%%%%%%%%%%%%%%%%%%%%%%%%%%%%%%%%%%%%%%%%%%%%%%%%
\begin{document}
\maketitle

In this document we declare who did what in our project. It is separated into
ideas, programming, literature search and the final article.

\section{Ideas}
Everyone in the group contributed equally to the following ideas, and all other ideas that were used and discussed throughout the group work.
\begin{itemize}
  \item Partial-order Planning
  \item Hierarchical Planning
  \item Goal Prioritisation
  \item Centralised Multibody Planner (high level planner)
\end{itemize}

\section{Programming}
We did our programming mostly in pairs. 
The first person is the one responsible and the second also contributed to the programming. 
The pairs did the debugging themselves.
First we declare the files and the people responsible followed by some bullet points declaring the relevant tasks and their authors.

\begin{itemize}
  \item \texttt{client.py} --- all
  \item \texttt{a\_star\_simple.py} --- \elias{} and \sune
  \item \texttt{convert\_path\_to\_moves.py} --- \jens{} and \superdude
  \item \texttt{parse\_lvl.py} --- \sune{} and \superdude
  \item \texttt{simple\_grid.py} --- \elias{} and \jens
  \item \texttt{high\_level\_plan.py} --- \elias{} and \sune
  \item \texttt{movable.py} --- \superdude{} and \jens
\end{itemize}

Now, we present the more specific tasks that were performed.    

\begin{itemize}
  \item Parse server messages (\sune{} and \superdude)
  \item Weighted A* search algorithm for shortest paths (\elias{} and \sune)
  \item Search algorithm for finding first free cell (not on desired path) (\elias{} and \sune)
  \item Convert path of cells to actions (parsable by server) (\jens{} and \superdude)
  \item Goal Prioritisation Technique (complex) (\jens)
  \item Goal Prioritisation Technique (simple) (\elias)
  \item Find closest box to given goal (\superdude)
  \item Find closest agent to box (\elias)
  \item Path resolving loop which was used to find the sub-path/original path on which the next set of movements should be carried out (\sune{} and \superdude)
  \item Represent level as a grid (\elias)
  \item Give boxes and agents default colors (\elias)
  \item Multibody planner (the high level planner) (\elias{} and \superdude)
  \item Partial-order Planner (discarded and unfinished) (\elias)
  \item A* search algorithm based on Java implementation from warmup (discarded and unused) (\sune)
\end{itemize}

\section{Litterature Search}
\begin{itemize}
  \item Articles on Partial-order planning (\sune)
  \item Heuristic (\elias)
  \item Goal Prioritisation articles (\jens)
  \item A*, weighted A*, Dijkstra's Algorithm, Floyd-Warshall algorithm (\elias)
  \item All Pairs Shortest Paths (\elias)
  \item Articles discussing how to work with sub-goals (\superdude)
\end{itemize}

\section{Article}
The entire article has been created and proof read by all group members.
For sections that have subsections we also declare the name of the author responsible for integrity of the entire section.
\begin{itemize}
  \item Abstract (\elias)
  \item Introduction (\elias)
    \begin{itemize}
      \item Background (\elias)
      \item Related work (\sune)
    \end{itemize}
  \item Methods (\elias{} and \sune)
    \begin{itemize}
      \item intro (\superdude)
      \item goal prioritisation (\jens)
      \item representing levels (\elias)
      \item constructing paths and searching (\elias)
      \item planning on the fly (\elias)
    \end{itemize}
  \item Results (\superdude)
    \begin{itemize}
      \item intro (\superdude)
      \item performance vs optimality (\elias)
      \item multi-agent vs single-agent levels (\jens)
    \end{itemize}
  \item Discussion (\jens)
    \begin{itemize}
      \item intro (\sune)
      \item online and offline planning (\elias)
      \item centralized and decentralized planning (\elias)
      \item goal ordering (\jens)
    \end{itemize}
  \item Conlusion (\sune)
    \begin{itemize}
      \item Future work (\elias{} and \jens)
    \end{itemize}
\end{itemize}

\end{document}
